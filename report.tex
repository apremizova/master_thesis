\documentclass[12pt]{article}

\usepackage[utf8]{inputenc}
\usepackage[english,russian]{babel}



\begin{document}
	
	\section*{Таблица с двоичными значениями}
	В таблице в каждой ячейке представлены сначала оптимальные (минимальные, т.к. ищем минимальную описательную длину) значения $\log_2{\frac{1}{\mu(x)}}=-(n_{01}\log_2{p}+n_{00}\log_2{(1-p)}+n_{10}\log_2{q}+n_{11}\log_2{(1-q)})$, затем $p$ и $q$, при которых оно достигается, представленные в двоичной системе счисления. По горизонтали отмечены значения $l$ - длина перебираемых $q$  в двоичной системе, по вертикали - значения $k$ - длина перебираемых $p$  в двоичной системе.
	
	\begin{table}[!h]
		\caption{Таблица оптимальных зн-й p и q в двоичной записи для $\pi$}
		\label{sometable}
		\begin{center}
			\begin{tabular}{|l|l|l|l|l|l|l|}
				\hline
				k / l &1 & 2 & 3 & 4 & 5 & 6\\
				\hline
				1 & 0.1& 0.1& 0.1& 0.1& 0.1& 0.1\\
				& 0.1& 0.10& 0.101& 0.1010& 0.10100& 0.100111\\
				& 25.0& 25.0& 24.4998& 24.4998& 24.4998& 24.4975\\
				\hline
				2 & 0.10& 0.10& 0.10& 0.10& 0.10& 0.10\\
				& 0.1& 0.10& 0.101& 0.1010& 0.10100& 0.100111\\
				& 25.0& 25.0& 24.4998& 24.4998& 24.4998& 24.4975\\
				\hline
				3 & 0.101& 0.101& 0.101& 0.101& 0.101& 0.101\\
				& 0.1& 0.10& 0.101& 0.1010& 0.10100& 0.100111\\
				& 24.8217& 24.8217& 24.3215& 24.3215& 24.3215& 24.3192\\
				\hline
				4 & 0.1001& 0.1001& 0.1001& 0.1001& 0.1001& 0.1001\\
				& 0.1& 0.10& 0.101& 0.1010& 0.10100& 0.100111\\
				& 24.7738& 24.7738& 24.2735& 24.2735& 24.2735& 24.2713\\
				\hline
				5 & 0.10011& 0.10011& 0.10011& 0.10011& 0.10011& 0.10011\\
				& 0.1& 0.10& 0.101& 0.1010& 0.10100& 0.100111\\
				& 24.7623& 24.7623& 24.2621& 24.2621& 24.2621& 24.2598\\
				\hline
				6 & 0.100101& 0.100101& 0.100101& 0.100101& 0.100101& 0.100101\\
				& 0.1& 0.10& 0.101& 0.1010& 0.10100& 0.100111\\
				& 24.7594& 24.7594& 24.2592& 24.2592& 24.2592& 24.2569\\
				\hline
			\end{tabular}
		\end{center}
	\end{table}
	
	\begin{table}[!h]
		\caption{Таблица оптимальных зн-й p и q в двоичной записи для $\pi$}
		\label{sometable}
		\begin{center}
			\begin{tabular}{|l|l|l|l|l|l|}
				\hline
				k / l &7 & 8 & 9 & 10 & 11\\
				\hline
				7 & 0.1001011& 0.1001011& 0.1001011& 0.1001011& 0.1001011\\
				& 0.1001111& 0.10011110& 0.100111011& 0.1001110110& 0.10011101100\\
				& 24.25487169& 24.25487169& 24.25474365& 24.25474365& 24.25474365\\
				\hline
				8 & 0.10010101& 0.10010101& 0.10010101& 0.10010101& 0.10010101\\
				& 0.1001111& 0.10011110& 0.100111011& 0.1001110110& 0.10011101100\\
				& 24.25469022& 24.25469022& 24.25456218& 24.25456218& 24.25456218\\
				\hline
				9 & 0.100101011& 0.100101011& 0.100101011& 0.100101011& 0.100101011\\
				& 0.1001111& 0.10011110& 0.100111011& 0.1001110110& 0.10011101100\\
				& 24.25464497& 24.25464497& 24.25451694& 24.25451694& 24.25451694\\
				\hline
				10 & 0.1001010101& 0.1001010101& 0.1001010101& 0.1001010101& 0.1001010101\\
				& 0.1001111& 0.10011110& 0.100111011& 0.1001110110& 0.10011101100\\
				& 24.25463365& 24.25463365& 24.25450561& 24.25450561& 24.25450561\\
				\hline
				11 & 0.10010101011& 0.10010101011& 0.10010101011& 0.10010101011& 0.10010101011\\
				& 0.1001111& 0.10011110& 0.100111011& 0.1001110110& 0.10011101100\\
				& 24.25463082& 24.25463082& 24.25450278& 24.25450278& 24.25450278\\
				\hline
			\end{tabular}
		\end{center}
	\end{table}
	
	Выводы к таблицам 1, 2: заметим, что при фиксированной длине l (по столбцам) двоичной записи переходной вероятности q оптимальное значение q неизменно, но при этом с увеличением k оптимальное значение логарифма уменьшается. Аналогично для фиксированного k (по строкам).
	
		\begin{table}[!h]
		\caption{Таблица оптимальных зн-й p и q в двоичной записи для $\sqrt{2}$}
		\label{sometable}
		\begin{center}
			\begin{tabular}{|l|l|l|l|l|l|l|}
				\hline
				k / l &1 & 2 & 3 & 4 & 5 & 6\\
				\hline
				1 & 0.1& 0.1& 0.1& 0.1& 0.1& 0.1\\
				& 0.1& 0.10& 0.101& 0.1010& 0.10100& 0.100111\\
				& 25.0& 25.0& 24.4998& 24.4998& 24.4998& 24.4975\\
				\hline
				2 & 0.10& 0.10& 0.10& 0.10& 0.10& 0.10\\
				& 0.1& 0.10& 0.101& 0.1010& 0.10100& 0.100111\\
				& 25.0& 25.0& 24.4998& 24.4998& 24.4998& 24.4975\\
				\hline
				3 & 0.101& 0.101& 0.101& 0.101& 0.101& 0.101\\
				& 0.1& 0.10& 0.101& 0.1010& 0.10100& 0.100111\\
				& 24.8217& 24.8217& 24.3215& 24.3215& 24.3215& 24.3192\\
				\hline
				4 & 0.1001& 0.1001& 0.1001& 0.1001& 0.1001& 0.1001\\
				& 0.1& 0.10& 0.101& 0.1010& 0.10100& 0.100111\\
				& 24.7738& 24.7738& 24.2735& 24.2735& 24.2735& 24.2713\\
				\hline
				5 & 0.10011& 0.10011& 0.10011& 0.10011& 0.10011& 0.10011\\
				& 0.1& 0.10& 0.101& 0.1010& 0.10100& 0.100111\\
				& 24.7623& 24.7623& 24.2621& 24.2621& 24.2621& 24.2598\\
				\hline
				6 & 0.100101& 0.100101& 0.100101& 0.100101& 0.100101& 0.100101\\
				& 0.1& 0.10& 0.101& 0.1010& 0.10100& 0.100111\\
				& 24.7594& 24.7594& 24.2592& 24.2592& 24.2592& 24.2569\\
				\hline
			\end{tabular}
		\end{center}
	\end{table}
	
	Выводы к таблице 3: для $\sqrt{2}$ то же, что и для $\pi$.
	
	\begin{table}[!h]
		\caption{Таблица оптимальных зн-й p и q в двоичной записи для $\sqrt{3}$}
		\label{sometable}
		\begin{center}
			\begin{tabular}{|l|l|l|l|l|l|l|}
				\hline
				k / l &1 & 2 & 3 & 4 & 5 & 6\\
				\hline
				1 & 0.1& 0.1& 0.1& 0.1& 0.1& 0.1\\
				& 0.1& 0.10& 0.100& 0.1001& 0.10001& 0.100010\\
				& 24.0& 24.0& 24.0& 23.9891& 23.9521& 23.9521\\
				\hline
				2 & 0.11& 0.11& 0.11& 0.11& 0.11& 0.11\\
				& 0.1& 0.10& 0.100& 0.1001& 0.10001& 0.100010\\
				& 21.9053& 21.9053& 21.9053& 21.8944& 21.8573& 21.8573\\
				\hline
				3 & 0.110& 0.110& 0.110& 0.110& 0.110& 0.110\\
				& 0.1& 0.10& 0.100& 0.1001& 0.10001& 0.100010\\
				& 21.9053& 21.9053& 21.9053& 21.8944& 21.8573& 21.8573\\
				\hline
				4 & 0.1100& 0.1100& 0.1100& 0.1100& 0.1100& 0.1100\\
				& 0.1& 0.10& 0.100& 0.1001& 0.10001& 0.100010\\
				& 21.9053& 21.9053& 21.9053& 21.8944& 21.8573& 21.8573\\
				\hline
				5 & 0.11001& 0.11001& 0.11001& 0.11001& 0.11001& 0.11001\\
				& 0.1& 0.10& 0.100& 0.1001& 0.10001& 0.100010\\
				& 21.8783& 21.8783& 21.8783& 21.8674& 21.8304& 21.8304\\
				\hline
				6 & 0.110010& 0.110010& 0.110010& 0.110010& 0.110010& 0.110010\\
				& 0.1& 0.10& 0.100& 0.1001& 0.10001& 0.100010\\
				& 21.8783& 21.8783& 21.8783& 21.8674& 21.8304& 21.8304\\
				\hline
			\end{tabular}
		\end{center}
	\end{table}
	
	Выводы к таблице 4: для $\sqrt{3}$ результаты уже отличаются от $\pi$, но наблюдаются те же закономерности.
	
\end{document}