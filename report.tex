\documentclass[12pt]{article}

\usepackage[utf8]{inputenc}
\usepackage[english,russian]{babel}



\begin{document}
	
	В таблице в каждой ячейке представлены сначала оптимальные значения $C(\mu)+\log_2{\frac{1}{\mu(x)}}$, затем $p$ и $q$, при которых оно достигается, округлённые до десятитысячных. По горизонтали отмечены значения $l$ - длина перебираемых $q$  в двоичной системе, по вертикали - значения $k$ - длина перебираемых $p$  в двоичной системе.
	
	\begin{table}[h]
		\caption{Таблица оптимальных значений p и q для $\pi$}
		\label{pitable}
		\begin{center}
		\begin{tabular}{|l|l|l|l|l|l|l|}
			\hline
			k / l &5 & 6 & 7 & 8 & 9 & 10\\
			\hline
			5 & 46.2582& 46.2559& 46.2546& 46.2546& 46.2545& 46.2545\\
			& 0.583& 0.583& 0.583& 0.583& 0.583& 0.583\\
			& 0.625& 0.6094& 0.6172& 0.6172& 0.6152& 0.6152\\
			
			\hline
			6 & 46.2582& 46.2559& 46.2546& 46.2546& 46.2545& 46.2545\\
			& 0.583& 0.583& 0.583& 0.583& 0.583& 0.583\\
			& 0.625& 0.6094& 0.6172& 0.6172& 0.6152& 0.6152\\
			\hline
			7 & 46.2582& 46.2559& 46.2546& 46.2546& 46.2545& 46.2545\\
			& 0.583& 0.583& 0.583& 0.583& 0.583& 0.583\\
			& 0.625& 0.6094& 0.6172& 0.6172& 0.6152& 0.6152\\
			\hline
			8 & 46.2582& 46.2559& 46.2546& 46.2546& 46.2545& 46.2545\\
			& 0.583& 0.583& 0.583& 0.583& 0.583& 0.583\\
			& 0.625& 0.6094& 0.6172& 0.6172& 0.6152& 0.6152\\
			\hline
			9 & 46.2582& 46.2559& 46.2546& 46.2546& 46.2545& 46.2545\\
			& 0.583& 0.583& 0.583& 0.583& 0.583& 0.583\\
			& 0.625& 0.6094& 0.6172& 0.6172& 0.6152& 0.6152\\
			\hline
			10 & 46.2582& 46.2559& 46.2546& 46.2546& 46.2545& 46.2545\\
			& 0.583& 0.583& 0.583& 0.583& 0.583& 0.583\\
			& 0.625& 0.6094& 0.6172& 0.6172& 0.6152& 0.6152\\
			\hline
		\end{tabular}
		\end{center}
	\end{table}
	
	
\end{document}