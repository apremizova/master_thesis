\documentclass[12pt]{article}

\usepackage[utf8]{inputenc}
\usepackage[english,russian]{babel}



\begin{document}
\title{Применение MDL (Minimal Descripton length) принципа Риссанена для полумарковских процессов.}
\author{Ремизова Анна Петровна}
\maketitle

\section*{Введение}
	Для начала рассмотрим простые марковские цепи. Пусть марковская цепь состоит из 2 состояний. Есть данные, мы хотим подобрать марковскую цепь, для которой наибольшая вероятность получить '001'*300. По Риссанену, если мы хотим предсказать, что будет дальше, то должны сравнивать друг с другом гипотезы по их сложности, причём даём преимущество простым гипотезам.
	
	$$C(\mu)+\log_2{\frac{1}{\mu(x)}}$$
	
	где $C(\mu)$ - complexity, $\mu$ - распределение вероятности.
	
\section*{Задача 1}
Дана последовательность состояний Марковской цепи из 2 состояний: 0 и 1. Найти оптимальные переходные вероятности $p$ из 0 в 1 и $q$ из 1 в 0 по принципу Риссанена MDL. 

Для решения этой задачи запишем вероятность получения заданной реализации: пусть $n_{ij}$ - число переходов из состояния $i$ в состояние $j$, тогда:

$$P_c(x) = p^{n_{01}}\cdot(i-p)^{n_{00}}\cdot q^{n_{10}}\cdot(1-q)^{n_{11}}\to max$$ 

$$\log_2{\frac{1}{P_c(x)}}=-(n_{01}\log_2{p}+n_{00}\log_2{(1-p)}+n_{10}\log_2{q}+n_{11}\log_2{(1-q)})$$

Сложность $C(\mu)$ будем определять как суммарную длину записи $p$ и $q$ в двоичной системе счисления. Пусть вероятность $p$ имеет $k$ знаков в двоичной системе, $q$ - $l$ знаков, тогда $C(\mu)=k+l$. Далее рассмотрим несколько реализаций Марковских цепей и исследуем, как меняются значения в зависимости от $k$ и $l$.

В таблице в каждой ячейке представлены сначала оптимальные значения $C(\mu)+\log_2{\frac{1}{\mu(x)}}$, затем $p$ и $q$, при которых оно достигается, округлённые до тысячных. По горизонтали отмечены значения $l$ - длина перебираемых $q$  в двоичной системе, по вертикали - значения $k$ - длина перебираемых $p$  в двоичной системе.

\begin{table}[h]
\caption{Таблица оптимальных значений p и q для $\pi$}
\label{pitable}
\begin{center}
\begin{tabular}{|l|l|l|l|l|l|l|}
	\hline
	k / l &5 & 6 & 7 & 8 & 9 & 10\\
	\hline
	5 & 46.258& 46.256& 46.255& 46.255& 46.255& 46.255\\
	& 0.583& 0.583& 0.583& 0.583& 0.583& 0.583\\
	& 0.625& 0.609& 0.617& 0.617& 0.615& 0.615\\
	\hline
	6 & 46.258& 46.256& 46.255& 46.255& 46.255& 46.255\\
	& 0.583& 0.583& 0.583& 0.583& 0.583& 0.583\\
	& 0.625& 0.609& 0.617& 0.617& 0.615& 0.615\\
	\hline
	7 & 46.258& 46.256& 46.255& 46.255& 46.255& 46.255\\
	& 0.583& 0.583& 0.583& 0.583& 0.583& 0.583\\
	& 0.625& 0.609& 0.617& 0.617& 0.615& 0.615\\
	\hline
	8 & 46.258& 46.256& 46.255& 46.255& 46.255& 46.255\\
	& 0.583& 0.583& 0.583& 0.583& 0.583& 0.583\\
	& 0.625& 0.609& 0.617& 0.617& 0.615& 0.615\\
	\hline
	9 & 46.258& 46.256& 46.255& 46.255& 46.255& 46.255\\
	& 0.583& 0.583& 0.583& 0.583& 0.583& 0.583\\
	& 0.625& 0.609& 0.617& 0.617& 0.615& 0.615\\
	\hline
	10 & 46.258& 46.256& 46.255& 46.255& 46.255& 46.255\\
	& 0.583& 0.583& 0.583& 0.583& 0.583& 0.583\\
	& 0.625& 0.609& 0.617& 0.617& 0.615& 0.615\\
	\hline
\end{tabular}
\end{center}
\end{table}

	
\tableofcontents
\end{document}