\documentclass[12pt]{article}

\usepackage[utf8]{inputenc}
\usepackage[english,russian]{babel}
\usepackage[left=2cm,right=2cm,top=2cm,bottom=2cm,bindingoffset=0cm]{geometry}
\usepackage{indentfirst}
\usepackage{comment}
\usepackage{pythonhighlight}

\usepackage[colorlinks, linkcolor = blue]{hyperref}


\newtheorem{sttm}{Утверждение}




\begin{document}
	\title{Применение MDL (Minimal Descripton length) принципа Риссанена для марковских процессов.}
	\author{Ремизова Анна Петровна}
	\maketitle
	
	\section*{Что нового}
	\begin{enumerate}
		\item Оказалось, что я неверно считала диграммы: я использовала в Pyhton метод строки \makebox{\pyth{s.count(substring)},} который считает непересекающиеся вхождения подстроки. Пример: для строки \pyth{'00000'} при подсчёте $n(00)$ данный метод выдавал ответ 2 при правильном ответе 4. Т.о., $n(00)$ и $n(11)$ занижались. Эти моменты я исправила и обновила таблицы (\ref{table:piBinary}, \ref{table:2Binary}, \ref{table:3Binary})
		\item Также в таблицах (\ref{table:piBinary}, \ref{table:2Binary}, \ref{table:3Binary}) в 1 строку каждой ячейки добавила значение MDL для данных $k, l$, на второй строке везде логарифм, затем $p, q$.
		\item Добавила Утверждение (\ref{sttm:independence}) про независимые оптимальные значения $p$ и $q$ и доказательство к нему.
		\item Добавила Таблицу (\ref{table:digramsPi23}) с количеством диграмм в рассмотренных случаях $pi, \sqrt{2}, \sqrt{3}$, а также (в процессе) оптимальные значения $\displaystyle k,l,\log{\frac{1}{\mu(x)}}, MDL$ к ним. При поиске оптимального значения уже учитывала тот факт, что $p$ и $q$ можно искать независимо друг от друга.
	\end{enumerate}
	
	\section*{Введение}
	 Есть данные, мы хотим подобрать марковскую цепь, для которой наибольшая вероятность получить заданную траекторию. По Риссанену, если мы хотим предсказать, что будет дальше, то должны сравнивать друг с другом гипотезы по их сложности, причём даём преимущество простым гипотезам. Выражения для Description length будет выглядеть следующим образом:
	
	\begin{equation}C(\mu)+\log_2{\frac{1}{\mu(x)}}\end{equation}
	
	где $C(\mu)$ -- complexity, $\mu$ -- распределение вероятности.
	
	\section*{Марковские цепи с 2 состояниями}
	Для начала рассмотрим простые марковские цепи. Пусть марковская цепь состоит из 2 состояний. Дана последовательность состояний Марковской цепи из 2 состояний: 0 и 1. Найти оптимальные переходные вероятности $p$ из 0 в 1 и $q$ из 1 в 0 по принципу Риссанена MDL. 
	
	Для решения этой задачи запишем вероятность получения заданной реализации: пусть $n(ij)$ -- число переходов из состояния $i$ в состояние $j$, тогда:
	
	\begin{equation}P_c(x) = p^{n(01)}\cdot(i-p)^{n(00)}\cdot q^{n(10)}\cdot(1-q)^{n(11)}\to max\end{equation}
	
	\begin{equation}\label{log}\log_2{\frac{1}{P_c(x)}}=-(n(01)\cdot\log_2{p}+n(00)\cdot\log_2{(1-p)}+n(10)\cdot\log_2{q}+n(11)\cdot\log_2{(1-q)})\end{equation}
	
	Сложность $C(\mu)$ будем определять как суммарную длину записи $p$ и $q$ в двоичной системе счисления. Пусть вероятность $p$ имеет $k$ знаков в двоичной системе, $q$ -- $l$ знаков, тогда $C(\mu)=k+l$. Далее рассмотрим несколько реализаций Марковских цепей и исследуем, как меняются значения в зависимости от $k$ и $l$.
	
	\subsection*{Таблица с двоичными значениями}
	В Таблицах~\ref{table:piBinary},~\ref{table:2Binary},~\ref{table:3Binary} в каждой ячейке представлены сначала оптимальные (минимальные, т.к. ищем минимальную описательную длину) значения $\log_2{\frac{1}{\mu(x)}}=-(n_{01}\log_2{p}+n_{00}\log_2{(1-p)}+n_{10}\log_2{q}+n_{11}\log_2{(1-q)})$, затем сложность по Риссанену, а после - значения $p$ и $q$, при которых оно достигается, представленные в двоичной системе счисления, для марковских цепей с траекториями, соответствующими 30 первым знакам $\pi, sqrt(2), sqrt(3)$ соответственно. По горизонтали отмечены значения $l$ - длина перебираемых $q$  в двоичной системе, по вертикали - значения $k$ - длина перебираемых $p$  в двоичной системе.
	
	
\begin{table}[h]
	\caption{Таблица оптимальных зн-й p и q в двоичной записи для $\pi$}
	\label{table:piBinary}
	\begin{center}
		\begin{tabular}{|l|l|l|l|l|l|l|}
			\hline
			k / l &1 & 2 & 3 & 4 & 5 & 6\\
			\hline
			1 & 32.0& 32.0& 32.0& 31.9891& 31.9521& 31.9521\\
			& 28.0& 28.0& 28.0& 27.9891& 27.9521& 27.9521\\
			& 0.1& 0.1& 0.1& 0.1& 0.1& 0.1\\
			& 0.1& 0.10& 0.100& 0.1001& 0.10001& 0.100010\\
			\hline
			2 & 34.0& 34.0& 34.0& 33.9891& 33.9521& 33.9521\\
			& 28.0& 28.0& 28.0& 27.9891& 27.9521& 27.9521\\
			& 0.10& 0.10& 0.10& 0.10& 0.10& 0.10\\
			& 0.1& 0.10& 0.100& 0.1001& 0.10001& 0.100010\\
			\hline
			3 & 36.0& 36.0& 36.0& 35.9891& 35.9521& 35.9521\\
			& 28.0& 28.0& 28.0& 27.9891& 27.9521& 27.9521\\
			& 0.100& 0.100& 0.100& 0.100& 0.100& 0.100\\
			& 0.1& 0.10& 0.100& 0.1001& 0.10001& 0.100010\\
			\hline
			4 & 37.9664& 37.9664& 37.9664& 37.9555& 37.9185& 37.9185\\
			& 27.9664& 27.9664& 27.9664& 27.9555& 27.9185& 27.9185\\
			& 0.1001& 0.1001& 0.1001& 0.1001& 0.1001& 0.1001\\
			& 0.1& 0.10& 0.100& 0.1001& 0.10001& 0.100010\\
			\hline
			5 & 39.9464& 39.9464& 39.9464& 39.9355& 39.8985& 39.8985\\
			& 27.9464& 27.9464& 27.9464& 27.9355& 27.8985& 27.8985\\
			& 0.10001& 0.10001& 0.10001& 0.10001& 0.10001& 0.10001\\
			& 0.1& 0.10& 0.100& 0.1001& 0.10001& 0.100010\\
			\hline
			6 & 41.9464& 41.9464& 41.9464& 41.9355& 41.8985& 41.8985\\
			& 27.9464& 27.9464& 27.9464& 27.9355& 27.8985& 27.8985\\
			& 0.100010& 0.100010& 0.100010& 0.100010& 0.100010& 0.100010\\
			& 0.1& 0.10& 0.100& 0.1001& 0.10001& 0.100010\\
			\hline
		\end{tabular}
	\end{center}
\end{table}
	
	
	Выводы к Таблице~\ref{table:piBinary} для $\pi$: заметим, что при фиксированной длине l (по столбцам) двоичной записи переходной вероятности q оптимальное значение q неизменно, но при этом с увеличением k оптимальное значение логарифма уменьшается. Аналогично для фиксированного k (по строкам).
	
	\begin{table}[h]
		\caption{Таблица оптимальных зн-й p и q в двоичной записи для $\sqrt{2}$}
		\label{table:2Binary}
		\begin{center}
			\begin{tabular}{|l|l|l|l|l|l|l|}
				\hline
				k / l &1 & 2 & 3 & 4 & 5 & 6\\
				\hline
				1 & 32.0& 32.0& 31.9148& 31.7965& 31.7965& 31.795\\
				& 28.0& 28.0& 27.9148& 27.7965& 27.7965& 27.795\\
				& 0.1& 0.1& 0.1& 0.1& 0.1& 0.1\\
				& 0.1& 0.10& 0.101& 0.1001& 0.10010& 0.100101\\
				\hline
				2 & 34.0& 34.0& 33.9148& 33.7965& 33.7965& 33.795\\
				& 28.0& 28.0& 27.9148& 27.7965& 27.7965& 27.795\\
				& 0.10& 0.10& 0.10& 0.10& 0.10& 0.10\\
				& 0.1& 0.10& 0.101& 0.1001& 0.10010& 0.100101\\
				\hline
				3 & 36.0& 36.0& 35.9148& 35.7965& 35.7965& 35.795\\
				& 28.0& 28.0& 27.9148& 27.7965& 27.7965& 27.795\\
				& 0.100& 0.100& 0.100& 0.100& 0.100& 0.100\\
				& 0.1& 0.10& 0.101& 0.1001& 0.10010& 0.100101\\
				\hline
				4 & 38.0& 38.0& 37.9148& 37.7965& 37.7965& 37.795\\
				& 28.0& 28.0& 27.9148& 27.7965& 27.7965& 27.795\\
				& 0.1000& 0.1000& 0.1000& 0.1000& 0.1000& 0.1000\\
				& 0.1& 0.10& 0.101& 0.1001& 0.10010& 0.100101\\
				\hline
				5 & 40.0& 40.0& 39.9148& 39.7965& 39.7965& 39.795\\
				& 28.0& 28.0& 27.9148& 27.7965& 27.7965& 27.795\\
				& 0.10000& 0.10000& 0.10000& 0.10000& 0.10000& 0.10000\\
				& 0.1& 0.10& 0.101& 0.1001& 0.10010& 0.100101\\
				\hline
				6 & 42.0& 42.0& 41.9148& 41.7965& 41.7965& 41.795\\
				& 28.0& 28.0& 27.9148& 27.7965& 27.7965& 27.795\\
				& 0.100000& 0.100000& 0.100000& 0.100000& 0.100000& 0.100000\\
				& 0.1& 0.10& 0.101& 0.1001& 0.10010& 0.100101\\
				\hline
			\end{tabular}
		\end{center}
	\end{table}
	
	Выводы к Таблице~\ref{table:2Binary}: для $\sqrt{2}$ то же, что и для $\pi$.
	
	\begin{table}[h]
		\caption{Таблица оптимальных зн-й p и q в двоичной записи для $\sqrt{3}$}
		\label{table:3Binary}
		\begin{center}
			\begin{tabular}{|l|l|l|l|l|l|l|}
				\hline
				k / l &1 & 2 & 3 & 4 & 5 & 6\\
				\hline
				1 & 32.0& 32.0& 32.0& 31.8419& 31.8419& 31.8419\\
				& 28.0& 28.0& 28.0& 27.8419& 27.8419& 27.8419\\
				& 0.1& 0.1& 0.1& 0.1& 0.1& 0.1\\
				& 0.1& 0.10& 0.100& 0.0111& 0.01110& 0.011100\\
				\hline
				2 & 32.9053& 32.9053& 32.9053& 32.7472& 32.7472& 32.7472\\
				& 26.9053& 26.9053& 26.9053& 26.7472& 26.7472& 26.7472\\
				& 0.11& 0.11& 0.11& 0.11& 0.11& 0.11\\
				& 0.1& 0.10& 0.100& 0.0111& 0.01110& 0.011100\\
				\hline
				3 & 34.9053& 34.9053& 34.9053& 34.7472& 34.7472& 34.7472\\
				& 26.9053& 26.9053& 26.9053& 26.7472& 26.7472& 26.7472\\
				& 0.110& 0.110& 0.110& 0.110& 0.110& 0.110\\
				& 0.1& 0.10& 0.100& 0.0111& 0.01110& 0.011100\\
				\hline
				4 & 36.8182& 36.8182& 36.8182& 36.6601& 36.6601& 36.6601\\
				& 26.8182& 26.8182& 26.8182& 26.6601& 26.6601& 26.6601\\
				& 0.1011& 0.1011& 0.1011& 0.1011& 0.1011& 0.1011\\
				& 0.1& 0.10& 0.100& 0.0111& 0.01110& 0.011100\\
				\hline
				5 & 38.8182& 38.8182& 38.8182& 38.6601& 38.6601& 38.6601\\
				& 26.8182& 26.8182& 26.8182& 26.6601& 26.6601& 26.6601\\
				& 0.10110& 0.10110& 0.10110& 0.10110& 0.10110& 0.10110\\
				& 0.1& 0.10& 0.100& 0.0111& 0.01110& 0.011100\\
				\hline
				6 & 40.8132& 40.8132& 40.8132& 40.6552& 40.6552& 40.6552\\
				& 26.8132& 26.8132& 26.8132& 26.6552& 26.6552& 26.6552\\
				& 0.101101& 0.101101& 0.101101& 0.101101& 0.101101& 0.101101\\
				& 0.1& 0.10& 0.100& 0.0111& 0.01110& 0.011100\\
				\hline
			\end{tabular}
		\end{center}
	\end{table}
	
	Выводы к Таблице~\ref{table:3Binary}: для $\sqrt{3}$ результаты уже отличаются от $\pi$, но наблюдаются те же закономерности. Отличие $\sqrt{3}$ от $\pi$ и $\sqrt{2}$ в количестве диграмм в их двоичной записи, были рассмотрены первые 30 знаков для каждого числа, не считая точки. Если для $\pi$ и $\sqrt{2}$ распределение количества диграмм близко к равномерному, то для $\sqrt{3}$ оно менее сбалансированно: количество диграмм 00 меньше остальных, а диграмм 11 - больше (см.Таблицу~\ref{table:digramsPi23}).
	
	\begin{sttm}\label{sttm:independence} Оптимальное значение p не зависит от q и наоборот, оптимальное значение q не зависит от p.
	\end{sttm}
	{\bf Доказательство.} Рассмотрим выражение (\ref{log}) для логарифма. Значения $n(00), n(01), n(10),n(11)$ -- постоянные, и данное выражения можно представить в виде линейной комбинации двух функций $f_1(p)+f_2(q)$. Соотвественно, при максимизации всего выражения (логарифм (\ref{log}) должен быть маленьким, а так как перед всем выражением стоит минус, то выражение в скобках должно быть большим), так как переменные $p$ и $q$ содержатся в отдельных слагаемых, необходимо найти минимум отдельно для $f_1(p)$ и $f_2(q)$, друг на друга их значения при минимизации не влияют. 
	
	\subsection*{Анализ диграмм}
	
	
	\begin{table}[!h]
		\caption{Числа, количество диграмм в них, оптимальные k и l}
		\label{table:digramsPi23}
		\begin{center}
			\begin{tabular}{|l|l|l|l|l|l|l|l|}
			\hline
			Число & $n(00)$ & $n(01)$ & $n(10)$ & $n(11)$ & k & l & $\log_2{\frac{1}{\mu(x)}}$\\
			\hline
			$\pi$ & 6 & 7 & 8 & 7 &&&\\
			\hline
			$\sqrt{2}$ & 7 & 7 & 8 & 6 &&&\\
			\hline
			$\sqrt{3}$ & 3 & 7 & 8 & 10 &&&\\
			\hline
		\end{tabular}
	\end{center}
	\end{table}


	\begin{comment}	\subsection*{Таблица с десятичными значениями}
	В таблице в каждой ячейке представлены сначала оптимальные значения $C(\mu)+\log_2{\frac{1}{\mu(x)}}$, затем $p$ и $q$, при которых оно достигается, округлённые до десятитысячных. По горизонтали отмечены значения $l$ - длина перебираемых $q$  в двоичной системе, по вертикали - значения $k$ - длина перебираемых $p$  в двоичной системе.
	
	\begin{table}[h]
		\caption{Таблица оптимальных значений p и q для $\pi$}
		\label{table:piDecimal}
		\begin{center}
			\begin{tabular}{|l|l|l|l|l|l|l|}
				\hline
				k / l &1 & 2 & 3 & 4 & 5 & 6\\
				\hline
				1 & 0.5& 0.5& 0.5& 0.5& 0.5& 0.5\\
				& 0.5& 0.5& 0.625& 0.625& 0.625& 0.6094\\
				& 25.0& 25.0& 24.4998& 24.4998& 24.4998& 24.4975\\
				\hline
				2 & 0.5& 0.5& 0.5& 0.5& 0.5& 0.5\\
				& 0.5& 0.5& 0.625& 0.625& 0.625& 0.6094\\
				& 25.0& 25.0& 24.4998& 24.4998& 24.4998& 24.4975\\
				\hline
				3 & 0.625& 0.625& 0.625& 0.625& 0.625& 0.625\\
				& 0.5& 0.5& 0.625& 0.625& 0.625& 0.6094\\
				& 24.8217& 24.8217& 24.3215& 24.3215& 24.3215& 24.3192\\
				\hline
				4 & 0.5625& 0.5625& 0.5625& 0.5625& 0.5625& 0.5625\\
				& 0.5& 0.5& 0.625& 0.625& 0.625& 0.6094\\
				& 24.7738& 24.7738& 24.2735& 24.2735& 24.2735& 24.2713\\
				\hline
				5 & 0.5938& 0.5938& 0.5938& 0.5938& 0.5938& 0.5938\\
				& 0.5& 0.5& 0.625& 0.625& 0.625& 0.6094\\
				& 24.7623& 24.7623& 24.2621& 24.2621& 24.2621& 24.2598\\
				\hline
				6 & 0.5781& 0.5781& 0.5781& 0.5781& 0.5781& 0.5781\\
				& 0.5& 0.5& 0.625& 0.625& 0.625& 0.6094\\
				& 24.7594& 24.7594& 24.2592& 24.2592& 24.2592& 24.2569\\
				\hline
			\end{tabular}
		\end{center}
	\end{table}}
	\end{comment}
	
	%\tableofcontents
\end{document}